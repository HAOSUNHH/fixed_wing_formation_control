%%==================================================
%% abstract.tex for BIT Master Thesis
%% modified by yang yating
%% version: 0.1
%% last update: Dec 25th, 2016
%%==================================================

\begin{abstract}

本文主要的研究内容为固定翼无人机紧密编队控制器设计及其应用。该控制器以固定翼无人机编队的领从方法(leader-follower method)为基础,并考虑现有
内环的姿态驾驶仪的控制输入量,完成从编队误差量到姿态控制输入量的计算,进而消除编队误差,达到编队目的。本文先定义编队控制误差,在进行控制器的设计。初步设计完成之后,使用MATLAB/Simulink等数学仿真工具研究控制器设计的稳定性以及动态特性。之后,再结合ROS/Gazebo等仿真工具,验证在考虑无人机的动力学模型、环境不确定性以及噪声的情况下的控制器的表现。
之后选取合适的无人机飞行平台,飞行控制硬件并编写移植编队控制算法,完成飞行实验验证。完成编队控制器的参数参数整定之后,并将实验的结果与仿真结果相对比,进行一定的改进。
最后,使用改进后的编队控制器完成双机编队任务,研究编队过程中的空气动力对于编队整体的飞行效率影响。
%TODO:这里的研究目的要针对研究情况改写。
%({\color{blue}{摘要是一篇具有独立性和完整性的短文,应概括而扼要地反映出本论文的主要内容。包括研究目的、研究方法、研究结果和结论等,特别要突出研究结果和结论。中文摘要力求语言精炼准确,硕士学位论文摘要建议500$\sim$800字,博士学位论文建议1000$\sim$1200字。摘要中不可出现参考文献、图、表、化学结构式、非公知公用的符号和术语。英文摘要与中文摘要的内容应一致。}})

\keywords{固定翼无人机、领从方法、紧密编队控制器设计、仿真、编队空气动力学、飞行实验}
%({\color{blue}{一般选3~8个单词或专业术语,且中英文关键词必须对应。})}}
\end{abstract}

\begin{englishabstract}

The main content of this thesis is to design the close formation controller adapted to the inner-loop attitude controller of the fixed-wing UAV, which is based on the leader-follower method of the UAV formation control. The close formation controller plays role of translating the formation errors to the input for the inner loop attitude controller.The formation control error is defined at the beginning of the thesis and then the formulations of the close formation controller. After the preliminary design of the formation controller, the MATLAB/Simulink is used to test and verify the dynamic quality and stability. Then considering the dynamic model of the  UAV, the environment uncertainty and the noise, the performance of the controller is verified with the ROS/Gazebo simulator. After that, the controller is rewrote to the algorithm running on the upper controller. The hardware of the controller and the UAV platform is well chosen to accomplish the formation experiment. During this period, the parameters are tuned in order to accomplish the optimal control effect.
Finally, the double fixed-wing UAV formation is conducted to verify the effect of the flight efficiency produced by the close formation.
   
\englishkeywords{fixed-wing UAV, leader-follower method, close formation controller design,\\simulation, formation aerodymatic, flight experiment}

\end{englishabstract}

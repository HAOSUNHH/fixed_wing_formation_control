%%
% The BIThesis Template for Bachelor Graduation Thesis
%
% 北京理工大学毕业设计(论文)结论 —— 使用 XeLaTeX 编译
%
% Copyright 2020 Spencer Woo
%
% This work may be distributed and/or modified under the
% conditions of the LaTeX Project Public License, either version 1.3
% of this license or (at your option) any later version.
% The latest version of this license is in
%   http://www.latex-project.org/lppl.txt
% and version 1.3 or later is part of all distributions of LaTeX
% version 2005/12/01 or later.
%
% This work has the LPPL maintenance status `maintained'.
%
% The Current Maintainer of this work is Spencer Woo.
%
% Compile with: xelatex -> biber -> xelatex -> xelatex

\chapter*{\vskip 10bp\textmd{结~~~~论} \vskip -6bp}
\addcontentsline{toc}{chapter}{结~~~~论}

% 在结论部分的子标题不需要序号,加上 * 即可(一个例子如下)
% \section*{结论段落标题}

% 这里插入一个参考文献,仅作参考
本文以固定翼无人机的紧密编队为研究对象,设计了一种符合现有开源无人机自动驾驶仪(PX4)姿态环输入的编队控制器。并利用MATLAB/Simulink对
无人机编队算法进行了运动学层面上的仿真验证;最后,使用ROS-Gazebo动力学仿真环境,对文中所设计的编队控制器进行了动力学层面上的
仿真。上述两种仿真的结果均显示此种控制器可以完成固定翼的紧密编队控制。

本文的主要研究结果如下:
\begin{enumerate}
    \item 明确了编队控制器设计需要使用的飞行力学以及导航坐标系,并在相应坐标系中将无人机的三维空间运动进行了简化,分为竖直以及水平
        两个通道分别控制;之后,按照简化的无人机运动,定义描述无人机编队的物理量;最后,建立了无人机运动的方程组。
    \item 以当下开源自动驾驶仪PX4为基础,介绍单架无人机的控制逻辑实现:包括导航模块、位置控制模块以及姿态控制模块的控制逻辑;最后介绍
        PX4的整理软件架构。
    \item 对于编队控制器进行设计:首先定义了编队误差的基本形式,根据控制的需要找出对应的控制输入。再利用增量式PID对于编队控制器进行控制;
        得到编队控制器的数学形式,将第一步完成的编队控制器期望值设计总能量控制器(TECS)进行再次转化,最终得到内环控制器对应的编队控制
        器的完整形式,编队控制器的输入为编队的位置误差,速度大小以及速度方向误差所定义的混合误差,控制器的输入是与姿态驾驶仪内环对应的
        无人机期望姿态以及期望推力。
    \item 对文中所得到的编队控制器利用MATLAB等工具进行数学仿真:利用文中提出的无人机编队的质点运动模型,不考虑无人机的姿态动力学过程,仅
        验证编队控制的控制逻辑的正确性。得到了相关的仿真实验数据以及结果图。
    \item 将文中提出的编队控制器利用C/C++编程语言,结合ROS-Gazebo仿真环境进行移植,进行工程实际应用时的优化。在考虑无人机的姿态动力学、运动学
        传感器噪声以及大气环境不确定性等因素的条件下,对于所提出的编队控制器进行动力学仿真。在此过程中,结合现有的PX4-QGroundControl地面站以及ROS
        所提供的rosbag等工具对编队过程进行记录和监视,经过参数调节以及优化,最终得到编队控制器在动力学仿真环境下的应用效果以及仿真。
\end{enumerate}
本文中所设计控制器还有很大的改进空间:首先,由于固定翼无人机速度的动力学惯性很大,尤其在速度大于期望速度的情况下,飞机减速的响应速度不佳,
在后续的设计改进之中应考虑固定翼无人机速度控制的特殊之处。其次,横侧向控制器中控制速度方向的部分优先级应该高于其他,但为了调整和参数
的方便性,因而将横侧向角度和位置误差进行混合。

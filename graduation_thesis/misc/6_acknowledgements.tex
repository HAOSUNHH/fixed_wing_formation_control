%%
% The BIThesis Template for Bachelor Graduation Thesis
%
% 北京理工大学毕业设计(论文)致谢 —— 使用 XeLaTeX 编译
%
% Copyright 2020 Spencer Woo
%
% This work may be distributed and/or modified under the
% conditions of the LaTeX Project Public License, either version 1.3
% of this license or (at your option) any later version.
% The latest version of this license is in
%   http://www.latex-project.org/lppl.txt
% and version 1.3 or later is part of all distributions of LaTeX
% version 2005/12/01 or later.
%
% This work has the LPPL maintenance status `maintained'.
%
% The Current Maintainer of this work is Spencer Woo.
%
% Compile with: xelatex -> biber -> xelatex -> xelatex

\chapter*{\vskip 10bp \textmd{致~~~~谢} \vskip -6bp}
\addcontentsline{toc}{chapter}{致~~~~谢}

正值本文完成之际,首先我想感谢母校北京理工大学能够给我这个机会让我可以完成毕业设计,为我的本科学习画上一个圆满的句号。

其次我想感谢的便是我的指导老师王佳楠老师。从初有固定翼编队控制的方向,到实验室固定翼项目组成立,再到本次毕业设计的完成,王老师以他
严谨的治学风格,高昂的学术热情,精益求精的工作作风,无论在学术之路还是人生发展方向都给了我莫大的感染与激励!无论是资金还是学习机会,
王老师总是尽力提供给我。在这些宝贵的学习机会之中,我对固定翼的控制的理解逐渐加深。
固定翼编队的项目,跌跌撞撞做了一年有余,正因为有成功时王老师给予我的赞扬,失利时给予我的鼓励与指导,此项目才得以到达今日。

系统与仿真实验室给了我一个如同家一样的地方,实验室中许多老师,例如丁艳老师,王春彦老师,王丹丹老师都曾经给过我许多帮助,并让我得以
实现心中所想,感谢系统与仿真实验室的一代代的建设者!在这之后,我要感谢亲自指导我的苏 劭署学长,是他将ROS-Gazebo这一
套系统第一次应用到了固定翼飞行平台之上;戚煜华、江佳奇师兄帮我解答了环境配置的相关问题并为我介绍了许多志同道合的人士;周正阳学长帮助我
解答了许多硬件问题以及通信的相关问题;陈亚东以及丁祥军学长解答了一系列的飞行器导航与制导的相关问题,在此一并致谢!固定翼小组在
成立之时,每个成员都为固定翼的控制贡献出了自己的力量与智慧;他们分别是孙浩、郑志强、赵亚明、罗正昕、鲁冰洁、刘哲伟同学以及2017
届学弟查家军、姜浩舸、张艺弘以及胡牧天学弟,感谢你们的并肩协力!

本项目还得到了国内阿木社区王根等几位前辈在硬件上的指导与帮助; 太原理工大学航模队杨明老师和他的团队以及荷兰代尔夫特理工大学王曦漫博士和他的团队帮
助我解决了固定翼多机编队仿真的环境问题,在此感谢上述前辈们的悉心指导!

此外,在前期的固定翼外场实验中,我的老搭档张智铎、范铮铮同学以及2015级航模队队长刘天学长、同届飞手教练孙黎明学长为在烈日酷暑下为此
项目实地实验,感谢你们的鼎力相助!


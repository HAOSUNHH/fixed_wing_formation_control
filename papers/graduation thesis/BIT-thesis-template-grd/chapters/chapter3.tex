\chapter{编队控制器设计}
\label{chap:controller_design}
第\ref{chap:formation_dynamic_equ}章中介绍了无人机双机编队的动力学模型,并将无人机运动分为铅垂平面以及水平平面;方程组\ref{fol_motion_eauation1}水平平面的直接输入量为$\Psi$的期望值,铅垂平面的
直接输入量为$\theta$和$V$的期望值。整体的控制逻辑框图如下图所示:
%TODO:此处加入整体的控制逻辑框图。
编队控制器的输入为定义的误差量,输出为无人机自动驾驶仪的内环输入值,即期望推力$T^{des}$,期望姿态$\Phi^{des},\theta^{des}$,偏航期望值$\Psi^{des}$将由内环姿态
自动驾驶仪按照协调转弯条件计算得到。本章的剩余部分将分别设计铅垂平面以及水平平面的控制器。
%TODO:此处的分配方式有些问题,考虑机体x和y
\section{水平平面编队控制器设计}
导航的本质是控制地速的方向,实现手段是产生垂直于速度方向的法向加速度;在无人机之中,多采用协调转弯(BTT)方式产生法向加速度。在导弹的制导规律之中,制导的最终目标是与期望的
点相交,而编队控制器的最终目标为:
\begin{enumerate}
    \item 从机速度方向与领机的速度方向一致。
    \item 从机的速度大小与领机的速度大小一致。
    \item 从机的位置与从机的期望位置一致。
\end{enumerate}
此处产生三种误差类型,这三类误差均投影在从机机体坐标系$O_bx_by_bz_b$之中,便于之后产生控制量:
\begin{enumerate}
    \item 领机与从机2维速度方向误差$\eta^{err}$。%TODO:记得改一下图,与之对应
    \item 领机与从机3维速度大小误差$(V_{X_b}^{err},V_{Y_b}^{err})$。
    \item 领机与从机3维位置误差$(P_{X_b}^{err},P_{Y_b}^{err},P_{Z_b}^{err})$
\end{enumerate}
因而此处水平平面的编队控制器的控制的任务是消除速度方向上的角度误差以及水平平面内的位置误差,即水平平面编队控制器应产生期望速度$V^{des}$以及期望法向加速度$a_{f}^{des}$。
\section{铅垂平面编队控制器设计}
